\makeglossaries

\newglossaryentry{apogee}
{
    name=apogee,
    description={The top of a rocket's trajectory or flight}
}

\newglossaryentry{drogue}
{
    name={drogue chute}
    description={A smaller chute which deploys at apogee, which lets the rocket fall fast enough to avoid excessive drift but slow enough for the main chute to safely deploy (usually about XXX m/s)}
}

\newglossaryentry{main}
{
    name={main chute}
    description={A larger chute which deploys 200-300 meters above the ground and slows the rocket enough to safely touch down.}
}

\newglossaryentry{epoxy}
{
    name={epoxy resin}
    description={An adhesive which combines two liquids (a resin and a plasticizer) to create an extremely tough plastic bond. Used in the creation of composite materials along with fibers of either glass or carbon.}
}

\newglossaryentry{eggtimer}
{
    name=eggtimer
    description={A proprietary (COTS) flight computer.}
}

\newglossaryentry{cg}
{
    name={center of gravity}
    description={The point on a rocket at which gravitational forces create no net moment. If you were to put your finger at this point, the rocket would balance perfectly on it.}
}

\newglossaryentry{cp}
{
    name={center of pressure}
    description={The point on a rocket where aerodynamic forces create no net moment with a very small angle of attack. If you have wind coming at a slight angle to the rocket, you would be able to stop it from rotating in space by putting your finger at this point.}
}

\newglossaryentry{yagi}
{
    name=yagi
    description={An antenna used for low-power telemetry.}
}

\newglossaryentry{bulkhead}
{
    name=bulkhead
    description={A plate spanning the width of a rocket which separates one section from another. Often used for connecting components, creating airtight seals, and distributing forces.}
}

\newglossaryentry{tender}
{
    name={Tender Descender}
    description={A COTS recovery device which uses an explosive charge to separate two linkages, breaking a line in half on-command.}
}

\newglossaryentry{raptor}
{
    name=Raptor
    description={A COTS recovery device which uses an explosive charge to puncture a canister of compressed gas.}
}

\newglossaryentry{motor}
{
    name=motor
    description={The part of the rocket which burns fuel to create thrust. Motors generally use solid fuel, and engines generally use liquid fuel.}
}

\newglossaryentry{shearpin}
{
    name={shear pin}
    description={A plastic screw which breaks with a relatively small force. Used in recovery to keep sections of the body together until deployment.}
}

\newglossaryentry{groundstation}
{
    name={ground station}
    description={The receiving end of the rocket's telemetry system. The Houston to our Apollo.}
}

\newglossaryentry{l1l2l3}

    name={L1, L2, L3}
    description={Classifications obtained by individuals in rocketry clubs. L1 (Level 1) allows you to buy motors rated H and I, L2 (Level 2) allows you to buy motors rated J and K, and L3 (Level 3) allows you to fly motors rated L and higher. To obtain a classification, you need to build a rocket, fly it, and successfully recover it.}
}

\newglossaryentry{stability}
{
    name=stability
    description={The distance (in calibers) between a rocket's center of gravity and center of pressure. An ideal range is 1.6-5.}
}

\newglossaryentry{caliber}
{
    name=caliber
    description={The body diameter of a rocket.}
}

\newglossaryentry{wetmass}
{
    name={wet mass}
    description={The mass of a rocket when fully fueled.}
}

\newglossaryentry{drymass}
{
    name={dry mass}
    description={The mass of a rocket with no fuel.}
}

\newglossaryentry{payload}
{
    name=Payload
    description={The team that designs payloads, or stuff in the rocket that doesn't affect the flight of the rocket itself. Scientific experiments, data collection, and mechanical experiments are all options for payloads.}
}

\newglossaryentry{structures}
{
    name=Structure
    description={The team which designs the airframe of the rocket.}
}

\newglossaryentry{recovery}
{
    name=Recovery
    description={The team which designs the mechanisms that return the rocket to the ground in one piece.}
}

\newglossaryentry{avionics}
{
    name=Avionics
    description={The team which designs the flight computers, which model the flight of the rocket in real time, transmit telemetry, and initiate pyro events.}
}

\newglossaryentry{pyro}
{
    name=pyro
    description={A flight computer event which deploys a parachute. Usually these deployments are done pyrotechnically, hence the name.}
}

\newglossaryentry{energetics}
{
    name=energetics
    description={The mechanism that actually releases the parachute. Often (but not always) done explosively.}
}

\newglossaryentry{fillet}
{
    name=fillet
    description={A thick, curved bead of epoxy resin running along an adhesive attachment. Often quite heavy, but massively increases the strength of such a connection.}
}

\newglossaryentry{}
{
    name=
    description=
}
\newacronym{srad}{SRAD}{Student Researched and Developed}

\newacronym{cots}{COTS}{Commercial Off-The-Shelf}

\newacronym{gfrp}{GFRP}{Glass Fiber Reinforced Polymer}

\newacronym{cfrp}{CFRP}{Carbon Fiber Reinforced Polymer}

\newacronym{lora}{LORA}{}

\newacronym{fea}{FEA}{Finite Element Analysis}

\newacronym{cad}{CAD}{Computer-Aided Design}

\newacronym{cfd}{CFD}{Computational Fluid Dynamics}

\newacronym{cg}{CG}{Center of Gravity}

\newacronym{cp}{CP}{Center of Pressure}

\newacronym{ukra}{UKRA}{United Kingdom Rocketry Association}